\documentclass[a4paper, 11pt]{article}
\usepackage{fancyhdr}
\usepackage{pagecounting}
\usepackage[margin=1in]{geometry}



\usepackage{fancybox}

\usepackage{graphicx}
\graphicspath{{figures/}}
\DeclareGraphicsExtensions{.pdf,.jpeg,.png}
\usepackage{amsmath}
%\usepackage{floatrow}
\usepackage{subfigure}
\usepackage{caption}
\DeclareCaptionType{copyrightbox}
%\usepackage{subcaption}

%\usepackage{verbatim}
\usepackage{fancyvrb}
\usepackage{color}
\usepackage{algorithmic}
\usepackage{array}
\usepackage{hyperref}
\usepackage{mdwmath}
\usepackage{mdwtab}
\usepackage{booktabs}
\usepackage{listings}
\usepackage{eqparbox}
%\usepackage{fixltx2e}
\usepackage{stfloats}
\usepackage{url}
\usepackage{comment}
\usepackage{framed}
\usepackage{pdfsync}
\usepackage{ifthen}
\usepackage{balance}
% call this with optional  [turnoff] to turn off notes and todos
\usepackage[turnoff]{notes}
%\usepackage{microtype}
\usepackage{multirow}
%\usepackage[]{biblatex}
%\newtheorem{definition}{Definition}
\usepackage{xspace}


\renewcommand{\baselinestretch}{0.97}


\newcommand{\cf}{\hbox{\emph{cf.}}\xspace}
\newcommand{\deletia}{\ldots [deletia] \ldots}
\newcommand{\etal}{\hbox{\emph{et al.}}\xspace}
\newcommand{\eg}{\hbox{\emph{e.g.,}}\xspace}
\newcommand{\ie}{\hbox{\emph{i.e.}}\xspace}
\newcommand{\st}{\hbox{\emph{s.t.}}\xspace}
\newcommand{\wrt}{\hbox{\emph{w.r.t.}}\xspace}
\newcommand{\viz}{\hbox{\emph{viz.}}\xspace}
\newcommand{\vs}{\hbox{\emph{v.s.}}\xspace}
\newcommand{\etc}{\hbox{\emph{etc.}}\xspace}

\newtheorem{theorem}{Theorem}[section]
\newcommand{\defref}[1]{Definition~\ref{#1}}
\newtheorem{definition}[theorem]{Definition}

\newcommand{\varname}[1]{{\small\texttt{#1}}\xspace}
%\newcommand{\tom}[1]{{\color{red} Tom: #1}\xspace}
\newcommand{\ch}[1]{{\color{black} #1}\xspace}
%\newfloatcommand{capbtabbox}{table}[][\FBwidth]

\newcommand{\gh}{GitHub\xspace}

\newcommand{\C}{{\tt C}\xspace}
\newcommand{\cpp}{{\tt C++}\xspace}
\newcommand{\csharp}{{\tt C\#}\xspace}
\newcommand{\objc}{{\tt Objective-C}\xspace}
\newcommand{\go}{{\tt Go}\xspace}
\newcommand{\java}{{\tt Java}\xspace}
\newcommand{\perl}{{\tt Perl}\xspace}
\newcommand{\py}{{\tt Python}\xspace}
\newcommand{\python}{{\tt Python}\xspace}
\newcommand{\php}{{\tt Php}\xspace}
\newcommand{\ruby}{{\tt Ruby}\xspace}
\newcommand{\js}{{\tt Java\-Script}\xspace}
\newcommand{\cs}{{\tt Coffee\-Script}\xspace}
\newcommand{\ts}{{\tt Type\-Script}\xspace}
\newcommand{\clojure}{{\tt Clojure}\xspace}
\newcommand{\erlang}{{\tt Er\-lang}\xspace}
\newcommand{\haskell}{{\tt Haskell}\xspace}
\newcommand{\scala}{{\tt Scala}\xspace}

\newcommand{\FDSM}{{\tt Functional-Dynamic-Strong-Managed}\xspace} 		
\newcommand{\FSSM}{{\tt Functional-Static-Strong-Managed}\xspace}  		
\newcommand{\PSSM}{{\tt Proc-Static-Strong-Managed}\xspace}         		
\newcommand{\PSWU}{{\tt Proc-Static-Weak-Unmanaged}\xspace}         		
\newcommand{\SDSM}{{\tt Script-Dynamic-Strong-Managed}\xspace}     		
\newcommand{\SDWM}{{\tt Script-Dynamic-Weak-Managed}\xspace}        		

\newcommand{\FDSMt}{{\tt Functional\--Dynamic\--Strong\--Managed}\xspace} 		
\newcommand{\FSSMt}{{\tt Functional\--Static\--Strong\--Managed}\xspace}  		
\newcommand{\PSSMt}{{\tt Proc Static\--Strong\--Managed}\xspace}         		
\newcommand{\PSWUt}{{\tt Proc Static\--Weak\--Unmanaged}\xspace}         		
\newcommand{\SDSMt}{{\tt Script\--Dynamic\--Strong\--Managed}\xspace}     		
\newcommand{\SDWMt}{{\tt Script\--Dynamic\--Weak\--Managed}\xspace}        		

%\newcommand{\pswu}{{procedural\-\_static\-\_weak\-\_unmanaged }\xspace}
%\newcommand{\pssm}{{procedural\-\_static\-\_strong\-\_managed }\xspace}
%\newcommand{\fssm}{{functional\_static\-\_strong\_manged }\xspace}
%\newcommand{\fdsm}{{functional\_dynamic\-\_strong\_manged }\xspace}

\newcommand{\fdsm}{{\tt Functional\--Dynamic\--Strong\--Managed}\xspace} 		
\newcommand{\fssm}{{\tt Functional\--Static\--Strong\--Managed}\xspace}  		
\newcommand{\pssm}{{\tt Proc\--Static\--Strong\--Managed}\xspace}         		
\newcommand{\pswu}{{\tt Proc\--Static\--Weak\--Unmanaged}\xspace}         		
\newcommand{\sdsm}{{\tt Script\--Dynamic\--Strong\--Managed}\xspace}     		
\newcommand{\sdwm}{{\tt Script\--Dynamic\--Weak\--Managed}\xspace}        		



\newcommand{\pd}[1]{{\scriptsize \todo{Prem:  {\color{blue} #1}}}}
\newcommand{\tableshrink}[1]{\vspace{-0.25in}}


%-------------------------------------------------------------------------------------------
% Research Questions and Results
%
\newcounter{RQCounter}

 %Pass in label for rq, then rq
%

\newcommand{\RQ}[2]{%
\refstepcounter{RQCounter} \label{#1}
	\vspace{0.1in} \noindent \textbf{RQ\arabic{RQCounter}.~#2 \vspace{0.05in}}
	
}

\newcommand{\RQBoxed}[2]{%
\refstepcounter{RQCounter} \label{#1}
\begin{framed}%
\filbreak
	%\noindent\textbf{Research Question \arabic{RQCounter}: }%
	\noindent\textbf{RQ\arabic{RQCounter}: }%
#2\end{framed}
}


% Pass in label for associated research question
%
\newcommand{\RS}[2]{%
\begin{framed}%
\filbreak
	%\noindent
\textbf{Result {\ref{#1}}:~}{\emph {#2}}%
\end{framed}
}



\newcommand\at{@}
\newcommand\lb{[}
\newcommand\rb{]}
\newcommand\Red[1]{\textcolor[rgb]{1.00,0.00,0.00}{\textbf{#1}}}
\newcommand\red[1]{\textcolor[rgb]{1.00,0.00,0.00}{#1}}
\newcommand\grayline[1]{\textcolor[rgb]{0.4,0.4,0.4}{{#1}}}
\newcommand\gray[1]{\colorbox[rgb]{0.8,0.8,0.8}{{#1}}}
\newcommand\brown[1]{\textcolor[rgb]{0.55,0.27,0.07}{#1}}
\newcommand\orange[1]{\textcolor[rgb]{1.00,0.55,0.30}{\textbf{#1}}}
%\newcommand\green[1]{\textcolor[rgb]{0.42,0.56,0.14}{#1}}
\newcommand\lime[1]{\textcolor[rgb]{0.13,0.55,0.13}{{#1}}}
\newcommand\blue[1]{\textcolor[rgb]{0.00,0.00,1.00}{{#1}}}
\newcommand\ColorLine[1]{\textcolor[rgb]{0.00,0.00,1.00}{{#1}}}
\newcommand\javagreen[1]{\textcolor[rgb]{0.25,0.5,0.35}{{#1}}}
\newcommand\javadocblue[1]{\textcolor[rgb]{0.25,0.35,0.75}{{#1}}}
\newcommand\dkgreen[1]{\textcolor[rgb]{0.0,0.6,0}{\textbf{#1}}}
\newcommand\green[1]{\textcolor[rgb]{0.0,0.6,0}{#1}}
\newcommand\mauve[1]{\textcolor[rgb]{0.58,0.0,0.82}{{#1}}}
\newcommand\yellow[1]{\textcolor[rgb]{1,0.9,0.5}{{\bf #1}}}
\newcommand\lblue[1]{\colorbox[rgb]{0.87,1,1}{{#1}}}

\newcommand\PYbg[1]{\textcolor[rgb]{0.00,0.50,0.00}{\textbf{#1}}}
\newcommand\PYbf[1]{\textcolor[rgb]{0.73,0.40,0.53}{\textbf{#1}}}
\newcommand\PYbe[1]{\textcolor[rgb]{0.40,0.40,0.40}{#1}}
\newcommand\PYbd[1]{\textcolor[rgb]{0.73,0.13,0.13}{#1}}
\newcommand\PYbc[1]{\textcolor[rgb]{0.00,0.50,0.00}{\textbf{#1}}}
\newcommand\PYbb[1]{\textcolor[rgb]{0.40,0.40,0.40}{#1}}
\newcommand\PYba[1]{\textcolor[rgb]{0.00,0.00,0.50}{\textbf{#1}}}
\newcommand\PYaJ[1]{\textcolor[rgb]{0.73,0.13,0.13}{#1}}
\newcommand\PYaK[1]{\textcolor[rgb]{0.00,0.00,1.00}{\textbf{#1}}}
\newcommand\PYaH[1]{\fcolorbox[rgb]{1.00,0.00,0.00}{1,1,1}{#1}}
\newcommand\PYaI[1]{\textcolor[rgb]{0.69,0.00,0.25}{#1}}
\newcommand\PYaN[1]{\textcolor[rgb]{0.00,0.00,1.00}{\textbf{#1}}}
\newcommand\PYaO[1]{\textcolor[rgb]{0.00,0.00,0.50}{\textbf{#1}}}
\newcommand\PYaL[1]{\textcolor[rgb]{0.73,0.73,0.73}{#1}}
\newcommand\PYaM[1]{\textcolor[rgb]{0.74,0.48,0.00}{#1}}
\newcommand\PYaB[1]{\textcolor[rgb]{0.00,0.25,0.82}{#1}}
\newcommand\PYaC[1]{\textcolor[rgb]{0.67,0.13,1.00}{#1}}
\newcommand\PYaA[1]{\textcolor[rgb]{0.00,0.50,0.00}{#1}}
\newcommand\PYaF[1]{\textcolor[rgb]{1.00,0.00,0.00}{#1}}
\newcommand\PYaG[1]{\textcolor[rgb]{0.10,0.09,0.49}{#1}}
\newcommand\PYaD[1]{\textcolor[rgb]{0.25,0.50,0.50}{\textit{#1}}}
\newcommand\PYaE[1]{\textcolor[rgb]{0.63,0.00,0.00}{#1}}
\newcommand\PYaZ[1]{\textcolor[rgb]{0.00,0.50,0.00}{\textbf{#1}}}
\newcommand\PYaX[1]{\textcolor[rgb]{0.00,0.50,0.00}{#1}}
\newcommand\PYaY[1]{\textcolor[rgb]{0.73,0.13,0.13}{#1}}
\newcommand\PYaR[1]{\textcolor[rgb]{0.10,0.09,0.49}{#1}}
\newcommand\PYaS[1]{\textcolor[rgb]{0.25,0.50,0.50}{\textit{#1}}}
\newcommand\PYaP[1]{\textcolor[rgb]{0.49,0.56,0.16}{#1}}
\newcommand\PYaQ[1]{\textcolor[rgb]{0.40,0.40,0.40}{#1}}
\newcommand\PYaV[1]{\textcolor[rgb]{0.00,0.00,1.00}{\textbf{#1}}}
\newcommand\PYaW[1]{\textcolor[rgb]{0.73,0.13,0.13}{#1}}
\newcommand\PYaT[1]{\textcolor[rgb]{0.50,0.00,0.50}{\textbf{#1}}}
\newcommand\PYaU[1]{\textcolor[rgb]{0.82,0.25,0.23}{\textbf{#1}}}
\newcommand\PYaj[1]{\textcolor[rgb]{0.00,0.50,0.00}{#1}}
\newcommand\PYak[1]{\textcolor[rgb]{0.73,0.40,0.53}{#1}}
\newcommand\PYah[1]{\textcolor[rgb]{0.63,0.63,0.00}{#1}}
\newcommand\PYai[1]{\textcolor[rgb]{0.10,0.09,0.49}{#1}}
\newcommand\PYan[1]{\textcolor[rgb]{0.67,0.13,1.00}{\textbf{#1}}}
\newcommand\PYao[1]{\textcolor[rgb]{0.73,0.40,0.13}{\textbf{#1}}}
\newcommand\PYal[1]{\textcolor[rgb]{0.25,0.50,0.50}{\textit{#1}}}
\newcommand\PYam[1]{\textbf{#1}}
\newcommand\PYab[1]{\textit{#1}}
\newcommand\PYac[1]{\textcolor[rgb]{0.73,0.13,0.13}{#1}}
\newcommand\PYaa[1]{\textcolor[rgb]{0.50,0.50,0.50}{#1}}
\newcommand\PYaf[1]{\textcolor[rgb]{0.25,0.50,0.50}{\textit{#1}}}
\newcommand\PYag[1]{\textcolor[rgb]{0.40,0.40,0.40}{#1}}
\newcommand\PYad[1]{\textcolor[rgb]{0.73,0.13,0.13}{#1}}
\newcommand\PYae[1]{\textcolor[rgb]{0.40,0.40,0.40}{#1}}
\newcommand\PYaz[1]{\textcolor[rgb]{0.00,0.63,0.00}{#1}}
\newcommand\PYax[1]{\textcolor[rgb]{0.60,0.60,0.60}{\textbf{#1}}}
\newcommand\PYay[1]{\textcolor[rgb]{0.00,0.50,0.00}{\textbf{#1}}}
\newcommand\PYar[1]{\textcolor[rgb]{0.10,0.09,0.49}{#1}}
\newcommand\PYas[1]{\textcolor[rgb]{0.73,0.13,0.13}{\textit{#1}}}
\newcommand\PYap[1]{\textcolor[rgb]{0.00,0.50,0.00}{#1}}
\newcommand\PYaq[1]{\textcolor[rgb]{0.53,0.00,0.00}{#1}}
\newcommand\PYav[1]{\textcolor[rgb]{0.00,0.50,0.00}{\textbf{#1}}}
\newcommand\PYaw[1]{\textcolor[rgb]{0.40,0.40,0.40}{#1}}
\newcommand\PYat[1]{\textcolor[rgb]{0.10,0.09,0.49}{#1}}
\newcommand\PYau[1]{\textcolor[rgb]{0.40,0.40,0.40}{#1}}



\begin{document}
\thispagestyle{fancy}
%\pagenumbering{gobble}
%\fancyhead[location]{text} 
% Leave Left and Right Header empty.
\lhead{}
\rhead{}
%\rhead{\thepage}
\fancyfoot[C]{\footnotesize \grayline{http://rayb.info}} 

%\pagestyle{myheadings}
%\markboth{Sundar Iyer}{Sundar Iyer}

\pagestyle{fancy}
\lhead{\grayline{\it Baishakhi Ray}}
\rhead{\grayline{\thepage/\totalpages{}}}
%\rhead{\thepage}
%\renewcommand{\headrulewidth}{0pt} 
%\renewcommand{\footrulewidth}{0pt} 
%\fancyfoot[C]{\footnotesize http://www.stanford.edu/$\sim$sundaes/application} 
%\ref{TotPages}

% This kind of makes 10pt to 9 pt.
\begin{small}

%\vspace*{0.1cm}
\begin{center}
{\large \bf RESEARCH STATEMENT}\\
\vspace*{0.4cm}
{\normalsize Baishakhi Ray}% (bairay@ucdavis.edu)}
\end{center}

In last few decades, software industry has grown enormously. According to a recent survey,
worldwide software industry has grown to \$407.3 Billion dollars~\cite{Gartner13} in 2013,
4.8\% more since the previous year. For maintaining such growth, it is very important to ensure
the quality of the software as well as the efficiency of the software development process. My
research focuses on achieving these goals.  

In particular, I work on {empirical software engineering}; by statistically analyzing large
scale software projects, both open source and proprietary, I understand the current software
engineering process. Then, leveraging the data-driven knowledge, I develop novel program
analysis algorithms and testing techniques to improve software quality as well as productivity
of software development process. Especially, my research has been able to detect more than a
hundred bugs and security issues in large scale software; my tools aid existing software
development process by recommending relevant code changes, implementing various bug prediction
tools, and automatically detecting ported code across different software projects.  


Following is a summary of my research contributions that address some of the fundamental
problems of software engineering.

\section*{\large Software Porting}

Changes in software development come in many forms. Some changes are frequent, idiomatic, or
repetitive (\eg adding checks for nulls or logging important values) while others are unique.
The repetitive changes require a significant amount of duplicate work and are often
error-prone~\cite{}.  To implement these changes, developers usually port code\textemdash copy
code from an existing implementation and adapt it to fit the new context. Though existing
literature has shown a general lack of uniqueness in the source code~\cite{}, the actual nature
and cost of porting are yet unknown. As part of my research, I investigated different aspects
(\eg extent, cost, quality \etc) of porting across different open source and proprietary
software including several Microsoft projects, Linux, and the BSDs. 

I find that developers port non-trivial amount of changes from both within and across projects.
Especially cross-system porting among forked software (a slightly different variant of an
existing project) incurs significant duplicate work. In a case study~\cite{Ray2012} across
FreeBSD, NetBSD, and OpenBSD, I confirmed that, in each BSD release, on average, more than
twelve thousand lines are ported, and more than 25\% active developers participate in that.
Similar trend is also observed for intra-system porting in Linux and the Microsoft projects.
These results show that the upkeep effort of porting is significant and call for new techniques
to automate porting.

As a first step towards porting automation, I develop a change recommendation system.  Learning
from previous code changes, the recommendation system recommends ported change in two phases:
(1) {\em change suggestion.} When developers select a code fragment to modify, this step
recommends possible changes that similar code has experienced previously, (2) {\em change
completion.} When developers port change, this step recommends other changes that co-occurred
with the ported change in the past. On average, this recommendation system suggests ported
changes with 52.11\% to 59.91\% precision, and recommend change completion correctly with
38.48\% to 42.95\% precision. This work is currently under review in a top Software Engineering
conference. 

To evaluate the quality of ported code, I also modeled and implemented a risk analysis system.
Learning from previously resolved bug locations, the risk analyzer predicts future defects. The
risk analyzer shows with statistical significance that ported changes are less defect prone
than regular changes. However, on a closure look I noticed that an incorrect adaptation of
ported code to the target context sometimes introduces {\em porting errors}. In order to
automatically detect such errors, I investigate: what are the common types of porting errors?
and how they can be automatically detected? I found five common categories of porting errors,
including incorrect control- and data-flow, code redundancy, inconsistent identifier renamings,
\etc Leveraging this categorization, I designed and implemented a static control- and
data-dependence analysis technique to detect and categorize porting inconsistencies and
successfully identified porting errors with 65\% to 73\% precision, an improvement by 14 to 17
percentage points \wrt previous tools~\cite{}. This work was nominated for distinguished paper
award in ASE'13 and is invited for special issue journal.

\section*{\large Data-driven Software Analysis}

Thanks to the large number of diverse open source projects available in software forges such as
\gh, it becomes possible to evaluate some long-standing questions about software engineering
practice. Each of these project repositories hosts source code along with entire evolution
history, description, mailing lists, bug database, \etc.  I and my colleagues implemented
number of code analyzer and text analysis tools to gather different metrics from \gh project
repositories. Then applying a series of advanced data analysis methods from machine learning,
statistical network analysis, visualization, and regression analysis techniques, I shed some
empirical light on some fundamental problems of software engineering. 

\begin{itemize}

  \item{\textbf{Effect of programming languages on software quality.}}
    This question has been a topic of much debate for a very long time. To answer, I
    gathered a very large data set from \gh (728 projects, 63 Million SLOC, 29,000 authors, 1.5
    million commits, in 17 languages). Using a mixed-methods approach, combining multiple
    regression modeling with visualization and text analytics, I studied the effect of language
    features such as static \vs dynamic typing, strong \vs weak typing on software quality. By
    triangulating findings from different methods, and controlling for confounding effects such
    as team size, project size, and project history, it is observed that language design does
    have a significant, but modest effect on software quality.  This work is published in FSE
    2014. It also received wild media coverage including Slashdot, Infoworld, Register, Reddit,
    Twitter \etc

  \item{\textbf{API Stability and Adoption in the Android Ecosystem.}}
    In today's software ecosystem, that is primarily governed by web, cloud and mobile
    technologies, APIs perform a key role to connect disparate software.  Big players like
    Google, FaceBook, Amazon constantly publish new APIs to accommodate new feature requests,
    bugs fixes, and performance improvement.  Our study shows that Android APIs are evolving at
    a fast rate of 116 API updates per month on average. However, client adoption of new APIs
    is rather slow.  Though clients tend to use rapidly evolving APIs most frequently, the rate
    of adoption lags by about 14 months.  This indicates developers tend to avoid new unstable
    APIs. To best of my knowledge, this is the first work studying Android API adoption and
    inform future studies on how to promote API adoption and ultimately facilitate the growth
    of software ecosystems. This work was published in ICSM 2013.

  
  \item{\textbf{Use of Assertions in \gh Projects.}}
     Asserts in a program are believed to help with automated verification, code
     understandability, maintainability, fault localization and diagnosis, all eventually
     leading to better software quality. Using a large dataset of asserts in C and C++
     programs, this claim was confirmed\textemdash methods with asserts do have significantly
     fewer defects. Asserts also appear to play a positive role in collaborative software
     development, when many programmers are working on the same method.  Use of asserts is
     further characterized along process and product metrics; developers add asserts to methods
     they have prior knowledge of, and of which they have greater ownership. Also, in a call
     graph network hub like methods (collector and dispenser of information) tend to have more
     asserts. Such detailed characterization of asserts would help to predict useful asserts in
     relevant locations.  This work is currently under submission at a premiere software
     engineering conference.

  \item{\textbf{Gender and Tenure Diversity in \gh Teams.}}
    Using \gh, I and my colleagues studied gender and tenure diversity in online programming teams.
    Using the results of a survey, and regression modeling of \gh data, we studied how diversity
    relates to team productivity and turnover. We show that both gender and tenure diversity are
    positive and significant predictors of productivity.  These results can inform decision making
    on all levels, leading to better outcomes in recruiting and performance.  This work is
    currently under submission. 
     
\end{itemize}


\section*{\large Automated Adversarial Testing}

Some pieces of software are designed to serve an identical purpose though they are part of
different products and are implemented in completely isolated environments. Such software
components may not share structural resemblance; however, they should be semantically
identical\textemdash given an input all of the similar components should produce similar
output.  Based on this philosophy, we build a differential testing
framework~\cite{KeemanDifferentialTesting1998} to test certificate validation implementations
of several popular SSL/TLS libraries including OpenSSL, GnuTLS, NSS, CyaSSL, GnuTLS, PolarSSL,
MatrixSSL, etc.

Modern network security rests on the Secure Sockets Layer (SSL) and Transport Layer Security
(TLS) protocols.  Distributed systems, mobile and desktop applications, embedded devices, and
all of secure Web rely on SSL/TLS for protection against network attacks.  This protection
critically depends on whether SSL/TLS clients correctly validate X.509 certificates presented
by servers during the SSL/TLS handshake protocol.  We design, implement, and apply the first
methodology for large-scale testing of certificate validation logic in SSL/TLS
implementations~\cite{brubaker2014using}.  Our first ingredient is {\em Frankencerts},
synthetic certificates that are randomly mutated from parts of real certificates and thus
include unusual combinations of extensions and constraints.  Our second ingredient is
differential testing: if one SSL/TLS implementation accepts a certificate while another rejects
the same certificate, we use the discrepancy as an oracle for finding flaws in individual
implementations.


Differential testing with frankencerts uncovered 208 discrepancies between popular SSL/TLS
implementations. Many of them are caused by serious security vulnerabilities.  For example, any
server with a valid X.509 version 1 certificate can act as a rogue certificate authority and
issue fake certificates for any domain, enabling man-in-the-middle attacks against MatrixSSL
and GnuTLS.  Several implementations also accept certificate authorities created by
unauthorized issuers, as well as certificates not intended for server authentication. We also
found serious vulnerabilities in how users are warned about certificate validation errors.
When presented with an expired, self-signed certificate, NSS, Safari, and Chrome (on Linux)
report that the certificate has expired\textemdash a low-risk, often ignored error\textemdash
but not that the connection is insecure against a man-in-the-middle attack. This work was
published in IEEE S\&P 2014 and won best practical paper award~\cite{brubaker2014using}.

\section*{\large Building Software Systems.}



\section*{Future Work}


\begin{comment}

Furthermore, once a feature is introduced in one of the BSDs, it suffers significant delay, on
average three releases, to propagate to other BSDs.


In summary, this study shows that the upkeep effort of maintaining forked variant products is
significant and calls for new techniques to automate porting. The findings of this study was
published in ACM SIGSOFT International Symposium on the Foundations of Software Engineering
(FSE 2012)~\cite{Ray2012}.


companies often have to
create product variant to satisfy conflicting customer requirements. A popular way to create
such variants is {\em forking}\textemdash copying an existing product to create a
slightly different variant. Once forked, the variant projects start evolving independently.
For example, FreeBSD, OpenBSD, and NetBSD were all evolved from the same code base, OpenSSH 
was forked from SSH,  LibreOffice from OpenOffice, etc.  In fact, due to increased availability 
of forking support in online software repositories like GitHub, forking has become truly 
pervasive\textemdash Linux kernel has $6,930$ forks in GitHub as of this article. 
However, despite its popularity, forking is often considered to be counter-productive in 
terms of increased maintenance effort~\cite{Raymond1999:CB}.
As multiple forked projects evolve in parallel, developers often need to port similar features
or bug fixes from one project to another, incurring duplicate maintenance effort. As a first
step to assess whether forking is a sustainable practice, I measure the extent and
characteristics of duplicate effort required to maintain the forked projects. I conducted an
in-depth case study involving 18 years of parallel development history of FreeBSD, NetBSD, and
OpenBSD, and confirmed that the overhead of porting across these forked projects is indeed
significant. In each BSD release, on average, more than twelve thousand lines are ported, and
more than 25\% active developers participate in that.  Furthermore, once a feature is
introduced in one of the BSDs, it suffers significant delay, on average three releases, to 
propagate to other BSDs. In summary, this study shows that the upkeep effort of maintaining 
forked variant products is significant and calls for new techniques to automate porting. The
findings of this study was published in ACM
SIGSOFT International Symposium on the Foundations of Software Engineering (FSE
2012)~\cite{Ray2012}.
Moreover, porting is not necessarily only performed across different systems. Developers often 
port code from one module to another, within the same project, to introduce similar feature or bug 
fixes. By studying two large projects (one from Microsoft and Linux), I find that on average 18\% to 25\%
of the code modifications are ported around inside the same project. Such repetitive changes are


Such knowledge can have several applications including code summarization, improving code review,
change recommendation, and automated program repair \etc For example, if ported changes are
recognized in a code review, then the developers who introduced the same change earlier can be
involved in the code review process. Conversely, unique changes could be highlighted to
guarantee that they are carefully reviewed.



companies often have to
create product variant to satisfy conflicting customer requirements. A popular way to create
such variants is {\em forking}\textemdash copying an existing product to create a
slightly different variant. Once forked, the variant projects start evolving independently.
For example, FreeBSD, OpenBSD, and NetBSD were all evolved from the same code base, OpenSSH 
was forked from SSH,  LibreOffice from OpenOffice, etc.  In fact, due to increased availability 
of forking support in online software repositories like GitHub, forking has become truly 
pervasive\textemdash Linux kernel has $6,930$ forks in GitHub as of this article. 

However, despite its popularity, forking is often considered to be counter-productive in 
terms of increased maintenance effort~\cite{Raymond1999:CB}.
As multiple forked projects evolve in parallel, developers often need to port similar features
or bug fixes from one project to another, incurring duplicate maintenance effort. As a first
step to assess whether forking is a sustainable practice, I measure the extent and
characteristics of duplicate effort required to maintain the forked projects. I conducted an
in-depth case study involving 18 years of parallel development history of FreeBSD, NetBSD, and
OpenBSD, and confirmed that the overhead of porting across these forked projects is indeed
significant. In each BSD release, on average, more than twelve thousand lines are ported, and
more than 25\% active developers participate in that.  Furthermore, once a feature is
introduced in one of the BSDs, it suffers significant delay, on average three releases, to 
propagate to other BSDs. In summary, this study shows that the upkeep effort of maintaining 
forked variant products is significant and calls for new techniques to automate porting. The
findings of this study was published in ACM
SIGSOFT International Symposium on the Foundations of Software Engineering (FSE
2012)~\cite{Ray2012}.

Moreover, porting is not necessarily only performed across different systems. Developers often 
port code from one module to another, within the same project, to introduce similar feature or bug 
fixes. By studying two large projects (one from Microsoft and Linux), I find that on average 18\% to 25\%
of the code modifications are ported around inside the same project. Such repetitive changes are
periodic, \ie developers introduce a change, then use it repeatedly for some period and then
stop using it.  Each developer usually has her own set of change patterns and developers use
them increasingly as they become more prolific.
Distinguishing such ported changes over other
changes is important, because such distinctions can have several possible applications in software
engineering:~\todo{reduce this}
 Risk analysis (One would expect that changes that are ported are less error prone),
Code reviews (If ported changes are recognized in a code review, then the developers who
introduced the same change earlier can be involved in the code review process), Recommendation
systems (ported changes can be used as input for recommendation systems), and : for example,
recommend how a line would typically be changed or after some change has been made, recommend
other changes that are typically made with the initial change based on past co-occurrence ({\em
change completion}), and automated program repair (ported changes in bug fixes typically are
better candidates for automated program repair operations than unique
changes~\cite{Nguyen-etal-13})

Since the repetitive work involved in porting is non-trivial, it is important to automate the
process. As a first step towards the automation, I develop a change recommendation system.
Learning from the previous changes, the recommendation system recommends change when (1)
developers select a code fragment to modify, it recommends possible changes that similar code
has experienced previously, (2) developers make a ported change, it recommends other change
patterns that co-occurred with that committed change in the past. On average, this
recommendation system suggests ported changes with 52.11\% to 59.91\% precision, and recommend
change completion correctly with 38.48\% to 42.95\% precision. This work is currently under
review in a top SE conference. 



During my study of cross-system porting in the BSDs, I also noticed that ported changes are in
general less error prone than the non-ported changes. Leveraging this knowledge, I implemented a
risk-analyzer and show that categorizing changes as ported and non-ported can further facilitate 
the risk assessment of change. The risk analyzer confirms with statistical significant ported
changes are less error-prone than non-ported changes.

Though less risky, porting sometimes introduce errors. During porting developers copy code 
from a reference implementation to a target implementation. 
The code in the reference often serves as a template, and then later adapted~\cite{Kim2004}.
In the prior study I have noticed that the process of adapting a change to fit another context can 
be error-prone, often resulting in {\em porting errors}. Such errors can take place
while porting both within or across systems. In order to automatically detect such porting
errors, I investigate (1) the types of porting errors found in practice, and (2) how to detect and 
characterize potential porting errors. Analyzing version histories, I found five categories of 
porting errors, including incorrect control- and data-flow, code redundancy, inconsistent identifier 
renamings, etc.  Leveraging this categorization, I design a static control- and data-dependence 
analysis technique to detect and categorize porting inconsistencies. My evaluation 
on code from four open-source projects shows that	the tool can detect porting inconsistencies with
  65\% to 73\% precision and 90\% recall, and identify inconsistency types with 58\% to 63\%
  precision and 92\% to 100\% recall.  In a comparison with two existing error detection tools,
  I improved precision by 14 to 17 percentage points. This work was published in ASE 2013 and
  nominated for distinguished paper award. It was also invited for journal submission.






\begin{comment}


\section*{\large Analyzing \& improving software quality}

\begin{itemize}

  \item{\textbf{Effect of programming languages on software quality.}}
A variety of debates ensue during discussions whether a given programming language is ``the
right tool for the job". While some of these debates can sometimes appear to be tinged with
almost religious fervor, most people would agree that a programming language has an effect not
only on the coding process but also on the properties of the resulting artifact. The size of
the effect can be significant and important overall, and can depend on the language properties.
Advocates of strong static typing argue that type inference will catch software bugs early.
Advocates of dynamic typing may argue that rather than spend a lot of time correcting annoying
static type errors arising from sound, conservative static type checking algorithms in
compilers, it's better to rely on strong dynamic typing to catch errors as and when they arise.
These debates, however, have largely been of the armchair variety; usually the evidence offered
in support of one position or the other tends to be anecdotal. 

Using the vast amount of open source code available  in \gh, I am planning to shed some
empirical light on the effect of language features such as static \vs dynamic typing, strong
\vs weak typing on software quality.  Empirical evidences for the associations,  if any,  that
exist between programming language choice, their properties, usage domains, and code quality,
can help developers make more educated choices. Also, such knowledge will guide language
designer to architect language features.
 
\item{\textbf{Detecting and Characterizing Semantic Inconsistencies in Ported Code.}}
  Adding similar features and bug fixes often requires porting program patches from reference
  implementations and adapting them to target implementations. Porting errors may result from
  faulty adaptations or inconsistent updates. This paper investigates (1) the types of porting
  errors found in practice, and (2) how to detect and characterize potential porting errors.
  Analyzing version histories, we define five categories of porting errors, including incorrect
  control- and data-flow, code redundancy, inconsistent identifier renamings, etc.  Leveraging
  this categorization, we design a static control- and data-dependence analysis technique, {\sc
  \bf SPA}, to detect and characterize porting inconsistencies. Our evaluation on code from
  four open-source projects shows that	 {\sc \bf SPA} can detect porting inconsistencies with
  65\% to 73\% precision and 90\% recall, and identify inconsistency types with 58\% to 63\%
  precision and 92\% to 100\% recall.  In a comparison with two existing error detection tools,
  {\sc \bf SPA} improves precision by 14 to 17 percentage points.

\item{\textbf{Automated Adversarial Testing of SSL/TLS Certificate Validation.}}
Some pieces of software are designed to serve an identical purpose though they are part of
different products and are implemented in completely isolated environments. Such software
components may not share structural resemblance; however, they should be semantically
identical\textemdash given an input all of the similar components should produce similar
output.  Based on this philosophy, we build a differential testing
framework~\cite{KeemanDifferentialTesting1998} to test certificate validation implementations
of several popular SSL/TLS libraries including OpenSSL, GnuTLS, NSS, CyaSSL, GnuTLS, PolarSSL,
MatrixSSL, etc.

Modern network security rests on the Secure Sockets Layer (SSL) and Transport Layer Security
(TLS) protocols.  Distributed systems, mobile and desktop applications, embedded devices, and
all of secure Web rely on SSL/TLS for protection against network attacks.  This protection
critically depends on whether SSL/TLS clients correctly validate X.509 certificates presented
by servers during the SSL/TLS handshake protocol.  We design, implement, and apply the first
methodology for large-scale testing of certificate validation logic in SSL/TLS
implementations~\cite{brubaker2014using}.  Our first ingredient is {\em Frankencerts},
synthetic certificates that are randomly mutated from parts of real certificates and thus
include unusual combinations of extensions and constraints.  Our second ingredient is
differential testing: if one SSL/TLS implementation accepts a certificate while another rejects
the same certificate, we use the discrepancy as an oracle for finding flaws in individual
implementations.


Differential testing with frankencerts uncovered 208 discrepancies between popular SSL/TLS
implementations. Many of them are caused by serious security vulnerabilities.  For example, any
server with a valid X.509 version 1 certificate can act as a rogue certificate authority and
issue fake certificates for any domain, enabling man-in-the-middle attacks against MatrixSSL
and GnuTLS.  Several implementations also accept certificate authorities created by
unauthorized issuers, as well as certificates not intended for server authentication. We also
found serious vulnerabilities in how users are warned about certificate validation errors.
When presented with an expired, self-signed certificate, NSS, Safari, and Chrome (on Linux)
report that the certificate has expired\textemdash a low-risk, often ignored error\textemdash
but not that the connection is insecure against a man-in-the-middle attack. This work was
published in IEEE S\&P 2014 and won best practical paper award~\cite{brubaker2014using}.

\item{\textbf{Assert Use in GitHub Projects.}}
  \emph{Asserts} have long been a strongly recommended (if non-functional) adjunct to programs.
They certainly don't add any user-evident feature value;  and  it can take quite some skill \&
effort to devise and add useful asserts. However, they are believed to add considerable value
to the developer.  Certainly, they can help with automated verification; but even in the
absence of that, claimed advantages include improved understandability, maintainability, easier
fault localization and diagnosis, all eventually leading to better software quality.  We focus
on this latter claim, and use a large dataset of asserts in C and C++ programs to explore the
connection between asserts and defect occurrence. Our data suggests a connection: methods with
asserts do have significantly fewer defects. This indicates that asserts do play an important
role in software quality; we therefore explored further the factors that play a role in
assertion placement: specifically, \emph{process factors} (such as developer experience and
ownership) and \emph{product factors}, particularly inter-procedural factors, exploring how the
placement of assertions in methods are influenced by local \& global network properties of the
callgraph. Finally, we also conduct a differential analysis of assertion use across different
application domains. 
\end{itemize}


\section*{\large Understanding evolution of software eco-system.}

\begin{itemize}

\item{\textbf{Cross-System Porting in Forked Projects.}}
Similar changes not-only take place within same project; similar products also implement
significant repetitive code either by {\it forking}\textemdash copying an existing product to
create a slightly different product, or {\it porting}\textemdash copying an existing feature
implementation or bug fix to another member of the same product family.  Some well known
product families are FreeBSD, OpenBSD, and NetBSD (evolved from the same codebase); OpenSSH and
SSH; LibreOffice and OpenOffice, etc. 

Software forking is often considered an ad hoc, low-cost alternative to principled product line
development~\cite{Raymond1999:CB}.  As a first step to assess whether forking is a sustainable
practice, we conducted an in-depth case study of 18 years of the BSD product family
history~\cite{Ray2012}. We comprehensively characterize cross-system porting in temporal,
spatial, and developer dimensions and find that maintaining forked projects involves
significant effort of porting patches from other projects. Cross-system porting happens
periodically and the porting rate does not necessarily decrease over time. A significant
portion of active developers participate in porting changes from peer projects. However, ported
changes are less defect-prone than non-ported changes. This study shows that the upkeep effort
of maintaining variant products is significant and calls for new techniques to automate
cross-system porting. This work was published in FSE 2012~\cite{Ray2012}.


\item{\textbf{An Empirical Study of API Stability and Adoption in the Android Ecosystem.}}
When APIs evolve to accommodate new feature requests, to fix bugs, to meet new standards, and
to provide higher performance, client applications often need to make corresponding changes to
use new or updated APIs~\cite{Al-Ekram2005:byaccident}.  Despite the benefits of new or
improved APIs, developer adoption is often slow among client applications.  For example,
Google's Android Operating System is evolving fast, yet the consumer pool is fragmented by the
Android version numbers and API adoption is slow. As a first step toward understanding the
impact of API evolution on software ecosystems, an in-depth case study was conducted of the
co-evolution behavior of Android API and dependent applications~\cite{mcdonnell2013empirical}. 

This study confirms that Android is evolving fast rate at a rate of 116 API updates per month
on average; Client adoption, however, is not catching up with the pace of API evolution. About
28\% of API references in client apps are outdated with a median lagging time of 16 months.
22\% of outdated API usages eventually upgrade to use newer API versions, but the propagation
time is much slower than the average interval of API release (3 months).  Fast evolving APIs
are used and adopted more by clients than slow evolving APIs. However, the average time taken
to adopt new versions is longer for fast evolving APIs. Further, API usage adaptation code is
more defect prone than the regular changes in client. 

This study confirms that Android application developers avoid API instability. However, such
slow adoption trend may pose various types of risks for client applications such as security
vulnerability or poor performance.  According to the American Civil Liberties Union (ACLU),
``the lag in software updates leaves smartphone users with out-of-date and dangerous
systems"~\cite{ACLU}.  ACLU filed complaints on such spotty Android updates, stating they could
potentially harm users by letting hackers steal user data by utilizing security holes. As a
part of future work, we would like to investigate more to identify factors affecting API
adoption, and its correlation with software reliability. The findings of this work is a crucial
first step and inform future studies on how to promote API adoption and ultimately facilitate
the growth of software ecosystems.

\end{itemize}


\section*{\large Improving productivity of software development process.}

\begin{itemize}

\item{\textbf{Automatic Change Recommendation System.}}  
%To create a new variant of an existing project, developers often copy an existing codebase and modify it. This process is called software forking. After forking software, developers often port new features or bug fixes from peer projects. Repertoire analyzes repeated work of cross-system porting among forked projects. It takes the version histories as in- put and identifies ported edits by comparing the content of individual patches. It also shows users the extent of ported edits, where and when the ported edits occurred, which de- velopers ported code from peer projects, and how long it takes for patches to be ported.
Software is built through many incremental changes over a period of time. Some changes are
frequent, idiomatic, or repetitive (e.g. adding checks for nulls or logging important values)
while others are unique. We hypothesize that unique changes are different from the more common
similar changes in important ways;  unique changes may require more expertise or represent code
that is more complex or prone to mistakes.  To test this hypothesis, we  develop a tool, {\em
Repertoire}, for identifying repetitive changes in a software project
history~\cite{Ray2012:RCP}.  {\em Repertoire} is the first automated tool for detecting
repetitive edits with high accuracy of 94\% precision and 84\% recall.  Based on the results of
applying {\em Repertoire} to Linux kernel and two large projects at
Microsoft~\cite{ray2014unique}, we find that on average 18\% to 25\% of the changes are
repetitive in a project and they are introduced periodically, \ie developers introduce a
change, then use it repeatedly for some period and then stop using it with an average lifetime
of 65 days. We also notice that developers have their own set of change patterns and developers
use them increasingly as they become more prolific. Distinguishing unique changes from the
repetitive ones has several possible applications in software engineering including code
summarization, risk analysis,  automated program repair etc.  As a proof of concept we
implement two such applications.

Repetitive changes can be used as input for recommendation systems: for example, recommend how
a line would typically be changed ({\em change recommendation}), or after some change has been
made, recommend other changes that are typically made with the initial change based on past
co-occurrence ({\em change completion}).  Our prototype implementation of such recommendation
system can recommend changes with up to 60\% precision and 67\% recall.

{\em Repertoire} is the first automated tool for detecting repetitive edits with high accuracy
of 94\% precision and 84\% recall.  Based on the results of applying {\em Repertoire} to Linux
kernel and two large projects at Microsoft~\cite{ray2014unique}, 


\item{\textbf{Risk analysis:}} We provide statistically significant empirical evidence that unique changes are 
more error prone than the changes that developers repeatedly make. Thus, developers should be careful when introducing unique changes.
Also, our risk analysis system estimate bug potential of a repetitive change pattern, based on its previous bug history. 
If the bug potential of a repetitive pattern is beyond a threshold, developers should take special care. 

\item{\textbf{Recommendation systems:}} 
\end{itemize}

Part of this work was published in FSE 2012~\cite{Ray2012:RCP, Ray2012} and rest is currently under review.







\section*{\large Other research} 
I also worked on other aspects of software engineering including 
an empirical study of API stability and adoption in the Android ecosystem~\cite{mcdonnell2013empirical}, 
and empirically analyzing the bugs that are fixed multiple times in Mozilla and Eclipse software~\cite{park2012empirical}.
I further built several systems software, \eg an operating system abstraction for efficiently managing GPUs~\cite{rossbach2011ptask}, 
 a protocol for secure and reliable covert channel~\cite{ray2008protocol}, and an ecosystem for Context-Aware Mobile Social Networks~\cite{beach2008whozthat, beach2009touch}, 

\vspace{0.5cm}


\section*{\large Ongoing  \& Future Research}

It has been observed that real
software, the kind programmers produce by the kLOC to solve
real-world problems, tends to  be ``natural", like speech or natural language; it  
tends to be highly repetitive, predictable,  and amenable to large-sample statistical methods. 
Recent work has exploited this \emph{naturalness of software}, to build
suggestion engines, porting tools, coding standards checkers, and idiom miners.  This 
phenomenon also inherently suggests that code that is surprising, or unpredictable is
suspicious; indeed, syntax errors have been found to be \emph{un}natural when using 
standard entropy measures. 
In this paper,
focus specifically on the language statistics of a large corpus of \emph{bug fix commits}, about
38,000, drawn from 10 different  projects, and evaluate the naturalness of buggy code, and whether
the fixing of these bugs increases the naturalness. We find that buggy code a) is more entropic (unnatural) 
b) this entropy reduces with fixing code and c) focusing on highly entropic lines is similar in cost-effectiveness 
to some well-known static bug finders (PMD, FindBugs). This work has two important implications: 
1) entropy may be a reasonable (language-independent, simpler) alternative way
to draw programmers' attention to problematic code to PMD or FindBugs;  2) search-based bug-fixing
methods may benefit from using entropy both for fault-localization and searching for
fixes. 



% I am enthusiastic to continue my work in software engineering, as well as to bridge security and software. 
% Some areas of future work include:


% \subsection*{\small Studying evolution of software ecosystem.}

% In a software ecosystem, when a related feature or bug fix is introduced in one project, it may need to be eventually ported to other projects.
% However, such patch propagation process is currently not very efficient. In our previous study of cross-system porting~\cite{Ray2012} we find that average time to port an edit from one system to another is more than 2 years. Sometimes it even takes up to 10 years for an edit to be propagated.
% I would like to build a notification system to promote fast and timely patching of applications. When a relevant edit is introduced in one project, all the related components of other projects may need to be notified. However, one cannot notify all the changes, as the forked projects usually evolve independently in different directions. Naively notifying all the changes will significantly increase the false positives. First, we have to find which components across the projects are closely related in terms of cross-system porting. Since ported edits are not confined within files having similar names or directory structures, establishing such relationships can be challenging. Prior porting history may be useful for identifying relevant components. 

% I would further like to investigate whether similar bug reports in two peer projects are fixed similarly. 
% Extending prior approaches to find duplicate bug reports~\cite{dupBugReport:Wang:ICSE2008,dupBugReport:Nguyen:ASE2012}, 
% we can find similar bug reports across different peer projects and extract their corresponding fixed patches. 
% Then, {\rep} can measure similarity between the fixed patches. This will help us to determine whether two similar bug reports in two related projects undergo similar fixes. If we confirm that a large number of fixes are actually similar, it would motivate building a patch recommendation system. When a new bug will be reported in a project, we will look for a similar bug report in the  sibling projects and suggest its fix. 

% \subsection*{\small Finding vulnerabilities in security sensitive software.} 
% I would like to continue my research on detecting security vulnerabilities using sophisticated software engineering tools. 
% One such project I am interested in is tracking the error code propagation in a software. As we have seen in SSL/TLS implementations, 
% clients often mask or ignore error code returned from APIs. I am interested to understand how these ignored error code affect security 
% properties of software. I further would like to implement a generic framework that can efficiently manage such security sensitive error code.

% \vspace*{1\baselineskip}
% \noindent
% In summary, I remain broadly interested in software engineering and would like to build tools to make future software better.

\end{comment}


\end{small}
\begin{footnotesize}
\bibliographystyle{abbrv}
\bibliography{res}
\end{footnotesize}

\end{document}

