\documentclass[a4paper, 10pt]{article}
\usepackage{fancyhdr}
\usepackage{pagecounting}
\usepackage[margin=1in]{geometry}



\usepackage{fancybox}

\usepackage{graphicx}
\graphicspath{{figures/}}
\DeclareGraphicsExtensions{.pdf,.jpeg,.png}
\usepackage{amsmath}
%\usepackage{floatrow}
\usepackage{subfigure}
\usepackage{caption}
\DeclareCaptionType{copyrightbox}
%\usepackage{subcaption}

%\usepackage{verbatim}
\usepackage{fancyvrb}
\usepackage{color}
\usepackage{algorithmic}
\usepackage{array}
\usepackage{hyperref}
\usepackage{mdwmath}
\usepackage{mdwtab}
\usepackage{booktabs}
\usepackage{listings}
\usepackage{eqparbox}
%\usepackage{fixltx2e}
\usepackage{stfloats}
\usepackage{url}
\usepackage{comment}
\usepackage{framed}
\usepackage{pdfsync}
\usepackage{ifthen}
\usepackage{balance}
% call this with optional  [turnoff] to turn off notes and todos
\usepackage[turnoff]{notes}
%\usepackage{microtype}
\usepackage{multirow}
%\usepackage[]{biblatex}
%\newtheorem{definition}{Definition}
\usepackage{xspace}


\renewcommand{\baselinestretch}{0.97}


\newcommand{\cf}{\hbox{\emph{cf.}}\xspace}
\newcommand{\deletia}{\ldots [deletia] \ldots}
\newcommand{\etal}{\hbox{\emph{et al.}}\xspace}
\newcommand{\eg}{\hbox{\emph{e.g.,}}\xspace}
\newcommand{\ie}{\hbox{\emph{i.e.}}\xspace}
\newcommand{\st}{\hbox{\emph{s.t.}}\xspace}
\newcommand{\wrt}{\hbox{\emph{w.r.t.}}\xspace}
\newcommand{\viz}{\hbox{\emph{viz.}}\xspace}
\newcommand{\vs}{\hbox{\emph{v.s.}}\xspace}
\newcommand{\etc}{\hbox{\emph{etc.}}\xspace}

\newtheorem{theorem}{Theorem}[section]
\newcommand{\defref}[1]{Definition~\ref{#1}}
\newtheorem{definition}[theorem]{Definition}

\newcommand{\varname}[1]{{\small\texttt{#1}}\xspace}
%\newcommand{\tom}[1]{{\color{red} Tom: #1}\xspace}
\newcommand{\ch}[1]{{\color{black} #1}\xspace}
%\newfloatcommand{capbtabbox}{table}[][\FBwidth]

\newcommand{\gh}{GitHub\xspace}

\newcommand{\C}{{\tt C}\xspace}
\newcommand{\cpp}{{\tt C++}\xspace}
\newcommand{\csharp}{{\tt C\#}\xspace}
\newcommand{\objc}{{\tt Objective-C}\xspace}
\newcommand{\go}{{\tt Go}\xspace}
\newcommand{\java}{{\tt Java}\xspace}
\newcommand{\perl}{{\tt Perl}\xspace}
\newcommand{\py}{{\tt Python}\xspace}
\newcommand{\python}{{\tt Python}\xspace}
\newcommand{\php}{{\tt Php}\xspace}
\newcommand{\ruby}{{\tt Ruby}\xspace}
\newcommand{\js}{{\tt Java\-Script}\xspace}
\newcommand{\cs}{{\tt Coffee\-Script}\xspace}
\newcommand{\ts}{{\tt Type\-Script}\xspace}
\newcommand{\clojure}{{\tt Clojure}\xspace}
\newcommand{\erlang}{{\tt Er\-lang}\xspace}
\newcommand{\haskell}{{\tt Haskell}\xspace}
\newcommand{\scala}{{\tt Scala}\xspace}

\newcommand{\FDSM}{{\tt Functional-Dynamic-Strong-Managed}\xspace} 		
\newcommand{\FSSM}{{\tt Functional-Static-Strong-Managed}\xspace}  		
\newcommand{\PSSM}{{\tt Proc-Static-Strong-Managed}\xspace}         		
\newcommand{\PSWU}{{\tt Proc-Static-Weak-Unmanaged}\xspace}         		
\newcommand{\SDSM}{{\tt Script-Dynamic-Strong-Managed}\xspace}     		
\newcommand{\SDWM}{{\tt Script-Dynamic-Weak-Managed}\xspace}        		

\newcommand{\FDSMt}{{\tt Functional\--Dynamic\--Strong\--Managed}\xspace} 		
\newcommand{\FSSMt}{{\tt Functional\--Static\--Strong\--Managed}\xspace}  		
\newcommand{\PSSMt}{{\tt Proc Static\--Strong\--Managed}\xspace}         		
\newcommand{\PSWUt}{{\tt Proc Static\--Weak\--Unmanaged}\xspace}         		
\newcommand{\SDSMt}{{\tt Script\--Dynamic\--Strong\--Managed}\xspace}     		
\newcommand{\SDWMt}{{\tt Script\--Dynamic\--Weak\--Managed}\xspace}        		

%\newcommand{\pswu}{{procedural\-\_static\-\_weak\-\_unmanaged }\xspace}
%\newcommand{\pssm}{{procedural\-\_static\-\_strong\-\_managed }\xspace}
%\newcommand{\fssm}{{functional\_static\-\_strong\_manged }\xspace}
%\newcommand{\fdsm}{{functional\_dynamic\-\_strong\_manged }\xspace}

\newcommand{\fdsm}{{\tt Functional\--Dynamic\--Strong\--Managed}\xspace} 		
\newcommand{\fssm}{{\tt Functional\--Static\--Strong\--Managed}\xspace}  		
\newcommand{\pssm}{{\tt Proc\--Static\--Strong\--Managed}\xspace}         		
\newcommand{\pswu}{{\tt Proc\--Static\--Weak\--Unmanaged}\xspace}         		
\newcommand{\sdsm}{{\tt Script\--Dynamic\--Strong\--Managed}\xspace}     		
\newcommand{\sdwm}{{\tt Script\--Dynamic\--Weak\--Managed}\xspace}        		



\newcommand{\pd}[1]{{\scriptsize \todo{Prem:  {\color{blue} #1}}}}
\newcommand{\tableshrink}[1]{\vspace{-0.25in}}


%-------------------------------------------------------------------------------------------
% Research Questions and Results
%
\newcounter{RQCounter}

 %Pass in label for rq, then rq
%

\newcommand{\RQ}[2]{%
\refstepcounter{RQCounter} \label{#1}
	\vspace{0.1in} \noindent \textbf{RQ\arabic{RQCounter}.~#2 \vspace{0.05in}}
	
}

\newcommand{\RQBoxed}[2]{%
\refstepcounter{RQCounter} \label{#1}
\begin{framed}%
\filbreak
	%\noindent\textbf{Research Question \arabic{RQCounter}: }%
	\noindent\textbf{RQ\arabic{RQCounter}: }%
#2\end{framed}
}


% Pass in label for associated research question
%
\newcommand{\RS}[2]{%
\begin{framed}%
\filbreak
	%\noindent
\textbf{Result {\ref{#1}}:~}{\emph {#2}}%
\end{framed}
}



\newcommand\at{@}
\newcommand\lb{[}
\newcommand\rb{]}
\newcommand\Red[1]{\textcolor[rgb]{1.00,0.00,0.00}{\textbf{#1}}}
\newcommand\red[1]{\textcolor[rgb]{1.00,0.00,0.00}{#1}}
\newcommand\grayline[1]{\textcolor[rgb]{0.4,0.4,0.4}{{#1}}}
\newcommand\gray[1]{\colorbox[rgb]{0.8,0.8,0.8}{{#1}}}
\newcommand\brown[1]{\textcolor[rgb]{0.55,0.27,0.07}{#1}}
\newcommand\orange[1]{\textcolor[rgb]{1.00,0.55,0.30}{\textbf{#1}}}
%\newcommand\green[1]{\textcolor[rgb]{0.42,0.56,0.14}{#1}}
\newcommand\lime[1]{\textcolor[rgb]{0.13,0.55,0.13}{{#1}}}
\newcommand\blue[1]{\textcolor[rgb]{0.00,0.00,1.00}{{#1}}}
\newcommand\ColorLine[1]{\textcolor[rgb]{0.00,0.00,1.00}{{#1}}}
\newcommand\javagreen[1]{\textcolor[rgb]{0.25,0.5,0.35}{{#1}}}
\newcommand\javadocblue[1]{\textcolor[rgb]{0.25,0.35,0.75}{{#1}}}
\newcommand\dkgreen[1]{\textcolor[rgb]{0.0,0.6,0}{\textbf{#1}}}
\newcommand\green[1]{\textcolor[rgb]{0.0,0.6,0}{#1}}
\newcommand\mauve[1]{\textcolor[rgb]{0.58,0.0,0.82}{{#1}}}
\newcommand\yellow[1]{\textcolor[rgb]{1,0.9,0.5}{{\bf #1}}}
\newcommand\lblue[1]{\colorbox[rgb]{0.87,1,1}{{#1}}}

\newcommand\PYbg[1]{\textcolor[rgb]{0.00,0.50,0.00}{\textbf{#1}}}
\newcommand\PYbf[1]{\textcolor[rgb]{0.73,0.40,0.53}{\textbf{#1}}}
\newcommand\PYbe[1]{\textcolor[rgb]{0.40,0.40,0.40}{#1}}
\newcommand\PYbd[1]{\textcolor[rgb]{0.73,0.13,0.13}{#1}}
\newcommand\PYbc[1]{\textcolor[rgb]{0.00,0.50,0.00}{\textbf{#1}}}
\newcommand\PYbb[1]{\textcolor[rgb]{0.40,0.40,0.40}{#1}}
\newcommand\PYba[1]{\textcolor[rgb]{0.00,0.00,0.50}{\textbf{#1}}}
\newcommand\PYaJ[1]{\textcolor[rgb]{0.73,0.13,0.13}{#1}}
\newcommand\PYaK[1]{\textcolor[rgb]{0.00,0.00,1.00}{\textbf{#1}}}
\newcommand\PYaH[1]{\fcolorbox[rgb]{1.00,0.00,0.00}{1,1,1}{#1}}
\newcommand\PYaI[1]{\textcolor[rgb]{0.69,0.00,0.25}{#1}}
\newcommand\PYaN[1]{\textcolor[rgb]{0.00,0.00,1.00}{\textbf{#1}}}
\newcommand\PYaO[1]{\textcolor[rgb]{0.00,0.00,0.50}{\textbf{#1}}}
\newcommand\PYaL[1]{\textcolor[rgb]{0.73,0.73,0.73}{#1}}
\newcommand\PYaM[1]{\textcolor[rgb]{0.74,0.48,0.00}{#1}}
\newcommand\PYaB[1]{\textcolor[rgb]{0.00,0.25,0.82}{#1}}
\newcommand\PYaC[1]{\textcolor[rgb]{0.67,0.13,1.00}{#1}}
\newcommand\PYaA[1]{\textcolor[rgb]{0.00,0.50,0.00}{#1}}
\newcommand\PYaF[1]{\textcolor[rgb]{1.00,0.00,0.00}{#1}}
\newcommand\PYaG[1]{\textcolor[rgb]{0.10,0.09,0.49}{#1}}
\newcommand\PYaD[1]{\textcolor[rgb]{0.25,0.50,0.50}{\textit{#1}}}
\newcommand\PYaE[1]{\textcolor[rgb]{0.63,0.00,0.00}{#1}}
\newcommand\PYaZ[1]{\textcolor[rgb]{0.00,0.50,0.00}{\textbf{#1}}}
\newcommand\PYaX[1]{\textcolor[rgb]{0.00,0.50,0.00}{#1}}
\newcommand\PYaY[1]{\textcolor[rgb]{0.73,0.13,0.13}{#1}}
\newcommand\PYaR[1]{\textcolor[rgb]{0.10,0.09,0.49}{#1}}
\newcommand\PYaS[1]{\textcolor[rgb]{0.25,0.50,0.50}{\textit{#1}}}
\newcommand\PYaP[1]{\textcolor[rgb]{0.49,0.56,0.16}{#1}}
\newcommand\PYaQ[1]{\textcolor[rgb]{0.40,0.40,0.40}{#1}}
\newcommand\PYaV[1]{\textcolor[rgb]{0.00,0.00,1.00}{\textbf{#1}}}
\newcommand\PYaW[1]{\textcolor[rgb]{0.73,0.13,0.13}{#1}}
\newcommand\PYaT[1]{\textcolor[rgb]{0.50,0.00,0.50}{\textbf{#1}}}
\newcommand\PYaU[1]{\textcolor[rgb]{0.82,0.25,0.23}{\textbf{#1}}}
\newcommand\PYaj[1]{\textcolor[rgb]{0.00,0.50,0.00}{#1}}
\newcommand\PYak[1]{\textcolor[rgb]{0.73,0.40,0.53}{#1}}
\newcommand\PYah[1]{\textcolor[rgb]{0.63,0.63,0.00}{#1}}
\newcommand\PYai[1]{\textcolor[rgb]{0.10,0.09,0.49}{#1}}
\newcommand\PYan[1]{\textcolor[rgb]{0.67,0.13,1.00}{\textbf{#1}}}
\newcommand\PYao[1]{\textcolor[rgb]{0.73,0.40,0.13}{\textbf{#1}}}
\newcommand\PYal[1]{\textcolor[rgb]{0.25,0.50,0.50}{\textit{#1}}}
\newcommand\PYam[1]{\textbf{#1}}
\newcommand\PYab[1]{\textit{#1}}
\newcommand\PYac[1]{\textcolor[rgb]{0.73,0.13,0.13}{#1}}
\newcommand\PYaa[1]{\textcolor[rgb]{0.50,0.50,0.50}{#1}}
\newcommand\PYaf[1]{\textcolor[rgb]{0.25,0.50,0.50}{\textit{#1}}}
\newcommand\PYag[1]{\textcolor[rgb]{0.40,0.40,0.40}{#1}}
\newcommand\PYad[1]{\textcolor[rgb]{0.73,0.13,0.13}{#1}}
\newcommand\PYae[1]{\textcolor[rgb]{0.40,0.40,0.40}{#1}}
\newcommand\PYaz[1]{\textcolor[rgb]{0.00,0.63,0.00}{#1}}
\newcommand\PYax[1]{\textcolor[rgb]{0.60,0.60,0.60}{\textbf{#1}}}
\newcommand\PYay[1]{\textcolor[rgb]{0.00,0.50,0.00}{\textbf{#1}}}
\newcommand\PYar[1]{\textcolor[rgb]{0.10,0.09,0.49}{#1}}
\newcommand\PYas[1]{\textcolor[rgb]{0.73,0.13,0.13}{\textit{#1}}}
\newcommand\PYap[1]{\textcolor[rgb]{0.00,0.50,0.00}{#1}}
\newcommand\PYaq[1]{\textcolor[rgb]{0.53,0.00,0.00}{#1}}
\newcommand\PYav[1]{\textcolor[rgb]{0.00,0.50,0.00}{\textbf{#1}}}
\newcommand\PYaw[1]{\textcolor[rgb]{0.40,0.40,0.40}{#1}}
\newcommand\PYat[1]{\textcolor[rgb]{0.10,0.09,0.49}{#1}}
\newcommand\PYau[1]{\textcolor[rgb]{0.40,0.40,0.40}{#1}}



\begin{document}
\thispagestyle{fancy}
%\pagenumbering{gobble}
%\fancyhead[location]{text} 
% Leave Left and Right Header empty.
\lhead{}
\rhead{}
%\rhead{\thepage}
\fancyfoot[C]{\footnotesize \grayline{http://rayb.info}} 

%\pagestyle{myheadings}
%\markboth{Sundar Iyer}{Sundar Iyer}

\pagestyle{fancy}
\lhead{\grayline{\it Baishakhi Ray}}
\rhead{\grayline{\thepage/\totalpages{}}}
%\rhead{\thepage}
%\renewcommand{\headrulewidth}{0pt} 
%\renewcommand{\footrulewidth}{0pt} 
%\fancyfoot[C]{\footnotesize http://www.stanford.edu/$\sim$sundaes/application} 
%\ref{TotPages}

% This kind of makes 10pt to 9 pt.
\begin{small}

%\vspace*{0.1cm}
\begin{center}
{\large \bf RESEARCH STATEMENT}\\
\vspace*{0.1cm}
{\normalsize Baishakhi Ray (bairay@ucdavis.edu)}
\end{center}


I am primarily interested in software engineering with a strong focus on empirical software engineering.
In particular, I analyze large scale software repositories for both open source and proprietary software to 
understand current software engineering practices. Based on this understanding, I develop novel program 
analysis techniques to make software more reliable and secure.

Modern software are developed using smaller building blocks, \eg  packages, libraries, APIs, methods, etc. 
Many of these building blocks often serve similar purpose. For example, OpenSSL, GnuTLS etc are 
different SSL libraries that implement Secure Socket Layer protocol. Even, software with different functionality 
may show syntactic resemblance. Gabel and Su~\cite{Gabel2010:uniqueness} and Hindle et al.\cite{Hindle:2012} 
showed that source code is in general repetitive and predictable in nature. My research primarily focuses on 
detecting and analyzing such structural and semantic similarities in evolving software. I also describe potential 
applications of leveraging software similarity and implement multiple applications, spanning from change 
recommendation system to bug detection tools. 

%\subsection*{Background}

\subsection*{\small Repetitiveness in Evolving Software.} Software is built through many incremental changes over a period of time
and come in many forms. Some changes are frequent, idiomatic, or repetitive (e.g. adding checks for nulls 
or logging important values) while others are unique. I hypothesize that unique changes are different from the more common similar 
changes in important ways;  unique changes may require more expertise or represent code that is more complex or prone to mistakes. 
To test this hypothesis, I  develop a tool, {\em Repertoire}, for identifying repetitive changes in a software project history. Based on the 
results of applying {\em Repertoire} to Linux kernel and two large projects at Microsoft, I find that on average 18\% to 25\% of the 
changes are repetitive in a project and they are introduced periodically, \ie developers introduce a change, then use it repeatedly for 
some period and then stop using it with an average lifetime of 65 days. I also notice that developers have their own set of change
patterns and developers use them increasingly as they become more prolific.  

Knowing which changes are unique and which changes are repetitive has several possible applications in software engineering: 
\begin{itemize}
\item \textbf{Risk analysis:} One would expect that changes that are
unique are more error prone than changes that developers repeatedly make. I
provide significant empirical evidence to support this statement. 

\item \textbf{Recommendation systems:} Repetitive changes can be used as input for 
recommendation systems: for example, recommend how a line would typically be changed 
({\em change recommendation}, or after some change has been made, recommend other repetitive changes 
that are typically made with the initial change based on past co-occurrence ({\em change completion}).
My prototype implementation of such recommendation system can recommend changes up to 60\% precision and 
67\% recall.

\item \textbf{Automated program repair:} 
Repetitive changes can also be leveraged for automated program repair operations. Nguyen et al.~\cite{nguyen2013study} 
provided initial empirical evidence for this hypothesis. They found that repetitive bug fixes are 
usually smaller changes and therefore automated patching tools could start with small changes and 
gradually compose them.

\end{itemize}







%
%There exists a rich body of previous work on code duplication~\cite{Kamiya2002,tairas,Cordy2011:largecloning}. However, detecting similar changes is not equivalent to detect 
%duplicate code clone, since not all code clones evolve in the same fashion, and similar changes are not restricted 
%to only cloned code. I showed that software is not only similar in structure they also change similarly.
%For this analysis, we develop a tool called {\em Repertoire} that compares the content and edit operations of program patches 
%to identify ported edits. Repertoire takes diff-based program patches at the release granularity as in- put
%
%Repertoireisanautomatedcross-systemportinganal- ysis tool, which finds ported edits with 94% precision and 84% recall. This tool can serve as a basis for as- sessing the extent and characteristics of cross-system porting among forked projects.
%


\subsection*{A Case Study of Cross-System Porting in Forked Projects}

\subsection*{An Empirical Study of API Stability and Adoption in the Android Ecosystem}

\subsection*{An Empirical Study of Supplementary Bug Fixes}

\subsection*{Detecting and Characterizing Semantic Inconsistencies in Ported Code}

\subsection*{Using Frankencerts for Automated Adversarial Testing of Certificate Validation in SSL/TLS Implementations}

\vspace{0.5cm}


\end{small}
\begin{footnotesize}
\bibliographystyle{abbrv}
\bibliography{res}
\end{footnotesize}

\end{document}

